% Main LaTeX source file for sound_effects.pdf

% Copyright © 2020 by John Sauter.
% Licensed under the Creative Commons Attribution-ShareAlike 4.0 International
% license.  See https://creativecommons.org/license/by-sa/4.0/.

\documentclass[letterpaper,twoside]{article}
\usepackage{fontspec}
\usepackage{amsmath}
\usepackage[english]{babel}
\usepackage{color}
\usepackage{multicol}
\usepackage{array}
\usepackage{longtable}
\usepackage{embedfile}
\usepackage{enumitem}
\usepackage[super]{nth}
\usepackage{fancyhdr}
\usepackage{xfrac}
\usepackage{fnpct}
\usepackage{siunitx}
\usepackage{graphics}
\usepackage{natbib}
\usepackage{hyperxmp}
\usepackage[tracking]{microtype}
\usepackage{hyphenat}
\usepackage{verbatim}
\usepackage[kpsewhich=true]{minted}
%\usepackage[usenames,dvipsnames]{color}

% Font choices: pick one.
% 1. Latin Modern matches Don Knuth's Computer Modern
%\setmainfont{Latin Modern Roman}[SmallCapsFont={Latin Modern Roman Caps}]
%\usepackage{unicode-math}

% 2. Libertine
\setmainfont[Ligatures={Common},Numbers=Proportional]{Linux Libertine O}
\setsansfont{Linux Biolinum O}
\usepackage{unicode-math}
\setmathfont[Scale=MatchUppercase]{libertinusmath-regular.otf}

% 3. Old Standard
%\setmainfont{Old Standard}[SmallCapsFont={Latin Modern Roman Caps}]

\usepackage[pdfencoding=unicode,pagebackref]{hyperref}
\bibliographystyle{plainnat}
\setcitestyle{numbers,square}
% Pages styles
%\setlength{\headheight}{22.5pt}
\pagestyle{fancy}
\fancyhead{}
\fancyhead[LE]{\thepage}
\fancyhead[CE]{{\scshape John Sauter}}
\fancyhead[CO]{{\scshape Sound Effects for Live Theatre}}
\fancyhead[RO]{\thepage}
\renewcommand{\headrulewidth}{0pt}
\fancyfoot{}
\setlength\tabcolsep{1mm}
\renewcommand\arraystretch{1.3}

\begin{document}
\embedfile[desc={Sound Effects for Live Theatre},
  mimetype={text/plain}]{sound_effects.tex}
\title{Sound Effects for Live Theatre\footnote{Copyright
    {\copyright} 2020 by John Sauter.
    This paper is made available under a
    Creative Commons Attribution-ShareAlike 4.0 International License.
    You can read a human-readable summary of the license at
    \url{https://creativecommons.org/licenses/by-sa/4.0}, which contains
    a link to the full text of the license.
    See also section \ref{section:Licensing} of this paper.}
}
\author{John Sauter\footnote{
    System Eyes Computer Store,
    20A Northwest Blvd.  Ste 345,
    Nashua, NH  03063-4066,
    e-mail: John\_Sauter@systemeyescomputerstore.com,
    telephone: (603) 424-1188}}
\hypersetup{unicode=true,
  pdfauthor={John Sauter},
  pdftitle={Sound Effects for Live Theatre},
  pdfsubject={Sound Effects for Live Theatre},
  pdfkeywords={Sound Effects, show\_control, sound\_effects\_player,
    Community Theatre},
  pdfcontactaddress={System Eyes Computer Store, 20A Northwest Blvd.  Ste 345},
  pdfcontactcity={Nashua},
  pdfcontactcountry={USA},
  pdfcontactemail={John\_Sauter@systemeyescomputerstore.com},
  pdfcontactphone={603-424-1188},
  pdfcontactpostcode={03063-4066},
  pdfcontactregion={New Hampshire},
  pdfcontacturl={https://www.systemeyescomputerstore.com},
  pdfcopyright={Copyright {\copyright} 2020 by John Sauter},
  pdfdate={2020-06-23},
  pdflicenseurl={https://creativecommons.org/licenses/by-sa/4.0},
  pdfmetalang={en-US}  
}
\date{2020-06-23}
\maketitle
\begin{abstract}
  How to produce sound effects for live theatrical performances
  on a small budget,
  for example for a Community Theatre production.
\end{abstract}
\begin{description}
\item[Keywords:]Sound Effects, Community Theatre.
\end{description}
  
%\tableofcontents
\newpage
\section{Purpose of Sound Effects}
The purpose of sound effects is to assist in telling the story.
That story is told mostly by the actors on the stage, augmented
by the costumes, props and scenery.  Just as the scenery should
not draw the audience's attention away from the story, neither
should the sound effects.

\section{Why Use a Computer}
Low-budget live musical theatre provides its non-professional singers
with microphones so they can be heard over the orchestra by the audience.
Even non-musicals generally need microphones for actors unless the
theater is very small or has excellent acoustics.
The audio mixer, speakers and amplifiers needed for
sound reinforcement can also be used for sound effects.

In the simplest cases, adding a source of recorded sound,
such as a CD player, with the sound effects as ``songs'' on the
player, will be enough.
This is adequate for background
sounds, but the timing of a music player is not precise enough
for a spot sound, such as a slap.
In some cases these spot sounds can be made by the actors
or a backstage person using objects, but for others, such as a gun
shot, this is undesirable.  In addition, background sounds are
severely constrained: if an automobile must start its engine
in response to a cue while the urban background sounds are playing,
you need a second CD player.

Computers have become inexpensive enough in recent years that it
is reasonable to consider using one in place of the CD players
during a live performance.  With suitable software, a computer
can play both background and spot sounds, and mix them together
as needed.

I will use an example to illustrate how to use the
sound\_effects\_player component
of the show\_control project to produce the sounds needed for
a live musical theatre production.

\section{What You Will Need}

To run the sound\_effects\_player you will need a computer running
the Fedora distribution of GNU/Linux.  The computer must also have
a sound card so it can output sound.  Sound\_effects\_player supports
up to eight channels of sound, but two are usually enough for Community
Theatre productions.  If you need more, USB sound players are not
very expensive.

To get the sound\_effects\_player software, download the source tarball
from github.  It is a component of the show\_control project, at this
URL: \url{https://github.com/ShowControl/sound\_effects\_player}.
Sound\_effects\_player also requires libtime, another component of
show\_control.  Details are in the {\ttfamily .spec} file.
Alternatively, you can install the sound\_effects\_player
from COPR under the name johnsauter / sound\_effects\_player.

Once you have the software installed you can look at the sample
configuration, the documentation and the first three examples
to learn about the
mechanics of the sound\_effects\_player and see how to make simple sounds.
In this paper I will assume that you have looked at that information
so I will concentrate on the high-level design of the sound
effects for a live performance.

\section{A Show with Complex Sound Effects}

Let us suppose that you have been asked to provide the sound effects
for a show with a challenging set of sound effects:
Children of Eden by Stephen Schwartz and John Caird.

In addition to the usual pre-show music and environment sounds
during each scene,
there are also a variety of spot sounds that must be played
when an on-stage action happens.
In addition, the director has challenged you to provide sound from
both the front and back of the audience, and has loaned you
the speakers from his home sound system to make this possible.
This includes a subwoofer that will be useful for the thunder.

The XML files developed in this paper are included with the distribution
of the sound\_effects\_player as example 4.

In order to make sounds come from behind the audience you have
positioned two self-powered speakers on stands
at the back of the theater along with the usual two at the front.
The subwoofer is under the stage, concealed from the audience
by a cloth drop from the front of the stage.

The music will be live but the actors are not professional singers,
so we will use body microphones on the actors so they can be heard
over the music.  These microphones go through the audio mixer and
the sound is sent equally to the two front speakers, so it seems
to the audience that it is coming from the center of the stage.

The sound effects enter five channels of the audio mixer from
the computer running sound\_effects\_player.  In order to get
five sound channels out of the computer you might need to use a USB
sound device such as the M-Audio Fast Track Ultra 8R or
the much newer StarTech 7.1 USB Audio Adapter Sound Card with
SPDIF Digital Audio.  You won't need a new sound output device,
of course, if your computer can already drive a 5.1 surround sound system.
The audio mixer routes the five sound channels from the computer
directly to the five speakers.  If you are using a 5.1 surround sound
output from the computer, the Front Center channel will be unused.

We wish to provide an easy-to-use interface for the sound effects
operator, so we will divide the show into discrete parts, and present
him with only the controls he needs in each part.  Theatre tradition
divides shows into acts and scenes, but Children of Eden has no scene
markers, so we we will make up our own based on changes in
the depicted location.
In addition, we also need some sounds before the first act and after the last.

I do not have permission from the owners of the copyright on
Children of Eden to post the script here, so I will ask you
to purchase a copy of the script from Music Theater International
at this URL: \url{https://www.mtishows.com/children-of-eden}
so you can follow along as I go through the script looking for
places that sound effects can help to tell the story.  When you are
creating sound effects for your own show you will have a copy of
the script as part of the right to produce the play.  Be sure to
talk to the director, in case he has some sound effects in mind.

\subsection{Before the House Opens}
You will want to test the sound system.  Before the house opens is the best
time to fix any cable failures or wiring errors.  This needn't be complex:
a single sound that identifies all five speakers in turn should be enough.
I have used a track called ``6-channel ID'' which identifies all of the
speakers in a 5.1 surround sound system.  You won't hear the front
center speaker; or if you do something is wired wrong.

\subsection{House is Open}
Play a medley of music from the play, perhaps recorded during rehearsal.
To avoid copyright issues with this paper, I have used a nice public-domain song
in example 4.  You won't have a copyright problem because as part of the right
to produce your play you have a right to play the music for your audience.

\subsection{Almost Ready for Curtain}
Your theater will probably want to make a safety announcement.
This might be done live by the management, but in case it isn't
offer the sound effects operator a choice of announcements
to play.  I have included in example 4 the announcements used at
the Souhegan High School in Amherst, NH, where I did the sound
for the Amherst PTA production of Children of Eden in 2010.
There are announcements for Morning, Evening and Afternoon
performances.  The Afternoon and Evening versions have
short, long and long humerous versions, whereas Morning
has only short.  You should record a voice familiar to your audience
making the announcements.

\subsection{Act 01 Scene 01 Page 001: Creation}
The play starts in the dark, before the Earth was created.
It isn't at all clear what sounds would convey this to the
audience, so we'll just have silence.

\subsection{Act 01 Scene 02 Page 007: Garden of Eden}
The Garden of Eden should sound like a garden full of mammals,
birds, insects and reptiles for Adam to name.  They should be
very soft until page 15, when they present themselves to Adam.
We want a sound different from the silence
of the void, but not loud enough to distract the audience
from the story.

\subsubsection{Act 01 Scene 02 Page 015: Animals Appear}
As the animals arrive their sounds become more noticable until
Father starts to sing.  The director will probably not want a
sound effect for this, but you never know.

\subsubsection{Act 01 Scene 02 Page 024: Tree of Knowledge}
The tree is said to be near a waterfall, so we can add the sound
of running water.  However, it is later depicted as also being in the
part of the garden where Adam and Eve live, so perhaps the
running water isn't needed, or maybe it should always be present.
It is easier to eliminate a sound
than to add one at the last minute, so we will have this sound
available, and not use it if the director doesn't want it.

I created the water sound in example 4 by obtaining four
stereo recordings of flowing water, and assigning them to
each pair of speakers around the room: front, rear, left and right.
Each speaker thus gets two channels.  The four stereo recordings
are of different lengths, so as they repeat the audience doesn't
hear the same sound every few seconds.

\subsubsection{Act 01 Scene 02 Page 033a: Eve Bites the Apple}
As Eve bites, make the sound of an apple being bitten into.

\subsubsection{Act 01 Scene 02 Page 033b: Eve Departs}
Eve walks off through the garden so
we can end the sound of running water.  Pressing any of the Stop
buttons marked ``Water'' stops all of them, leaving one Start button
active.

\subsubsection{Act 01 Scene 02 Page 040: Adam Bites the Apple}
Make the apple biting sound again.

\subsubsection{Act 01 Scene 02 page 042: Thunder and Lightning}
After Father says ``forever will it burn!'' there is a terrific
crash of thunder as a bolt of lightning strikes the Tree of Knowledge.
This should be loud.  Use all four speakers to good effect
starting with those at the front.  I used Audacity to move the sound
from front to rear.  In addition, to enhance the bass
I filtered a monophonic version of
the sound to eliminate all frequencies over 100 Hz and fed that
to the LFE1 speaker, under the stage.  The Audacity project is in
example 4 under Weather.

\subsection{Act 01 Scene 03 Page 043: The Wasteland}
We transition to a new scene, where the background sound
is not of a garden but of a wasteland.  There is a waterfall
nearby, so we will also have the sound of running water, available,
as we did in the Garden of Eden.

\subsubsection{Act 01 Scene 03 Page 044: Birth of Cain and Abel}
Use the cry of a newborn baby for each birth.  Label each cluster
with the name of the baby.

\subsubsection{Act 01 Scene 03 Page 061: Eve slaps Cain}
The slap is a spot sound.

\subsubsection{Act 01 Scene 03 Page 062: Circle of Giant Standing Stones}
To convey that this is a diffrerent place the wasteland sound
should be different, perhaps with more life in it.  I have added
the Garden of Eden as additional background.

\subsubsection{Act 01 Scene 03 Page 065: Distant Rumble of Thunder}
Use the rear speakers only, to convey that the thunder is distant.

\subsubsection{Act 01 Scene 03 Page 069: Adam Strikes Cain}
We can use the same slap as when Eve slapped Cain, though louder.

\subsubsection{Act 01 Scene 03 Page 070: Adam Strikes Cain Again}
Again use the same slap.

\subsubsection{Act 01 Scene 03 Page 071: Cain Kills Abel}
Use a different sound for the blows that Cain strikes.

\subsection{Act 01 Scene 04 Page 072: The Mark of Cain}
As Adam and Eve exit with Abel's body, we transition to a new
scene.  The background sounds are wind and rain.  They are
on separate clusters so the sound effects operator can adjust
them as needed, and they slowly fade in when he starts them.

When Father first calls out to Cain there is thunder from
all four speakers, but it ends just before Father calls out
a second time.

\subsubsection{Act 01 Scene 04 Page 073: Cain is Marked}
When Father and Storytellers sing
``the race of Cain must ever bear this mark!''
there is immediately a flash of lightning and a loud crash of thunder.
I know that thunder is supposed to follow lightning, but I did not
want to wait for the lighting guys so I told them that I had rehearsed
the thunder and knew just when it should sound, and they should do
their best to keep up.
When Cain exits and Eve enters we transition to a new scene,
back in the wilderness.  The wind and rain fade out for five seconds.

\subsection{Act 01 Scene 05 Page 073: Eve Enters}
Use the same wilderness background as for scene 3.

\subsection{Intermission}
There is no background sound during intermission, but at its end
there is an optional safety announcement
as the audience is taking their seats.  I have included the one
from the Souhegan High School in Amherst, NH.

\subsection{Act 02 Scene 01 Page 081: Generations of Adam}
The background should be of a nice place, with a hint of a distant
stream and the occasional insect.  It is not as alive as the garden,
but not as dead as the wasteland.

\subsubsection{Act 02 Scene 01 Page 91: Noah Pounds}
Thud as Noah pounds in the last peg of gopher wood.
This is probably better done live.

\subsubsection{Act 02 Scene 01 Page 101: Animals Appear}
Each animal has an introductory sound: bunnies, turtles, chimps,
panthers, birds, frogs, mike, zebras, ostriches, giraffes and elephants.
Since their order is well-defined, we can present each to the sound
effects operator when the previous one has completed.  This lets us
use the same button for each sound--the sound effects operator
just presses ``Start'' when the next animal appears.

\subsubsection{Act 02 Scene 01 Page 103: Thunder}
The script calls for just one thunder on this page, but the director
has asked for four: a very distant, very bass rumble at the top
of the page, a closer rumble at ``not a stranger to the rain'',
closer yet at ``sacred and profane'', and even closer at ``let it rain'',
which is where the script calls for thunder.

\subsubsection{Act 02 Scene 01 Page 107: Thunder}
As Japheth and Yonah kiss, there is a close rumble of thunder.
Use all four speakers, though mostly in the rear.
When Japheth says ``come on!'' there is a louder and closer
thunder clap.  Use front and rear speakers equally.

\subsection{Act 02 Scene 02 page 108: Rain}
The rain starts, mostly in the front speakers.  This starts a new scene.

\subsubsection{Act 02 Scene 02 Page 109: Thunder}
On ``Hurricane'' the rain starts.
After ``forever will it rain'' there is a flash of lightning
and a big clap of thunder.  Start it in the rear speakers
and move it quickly to the front.
After ``and now I feel so old'' there is another thunder.

\subsection{Act 02 Scene 03 Page 110: On the Deck of the Ark}
There is rain falling onto the gopher-wood ark.
The rain is mostly in the front speakers.
There is also wind, using all four speakers.

\subsubsection{Act 02 Scene 03 Page 114: Bird Call}
There is a bird sound from a birdcage.  Put this in the speaker
closest to the birdcage.

\subsubsection{Act 02 Scene 03 Page 116: Commotion of People and Animals}
Use the front speakers only.

\subsubsection{Act 02 Scene 03 Page 118: Noah Strikes Japheth}
Use the same slap as in act 1 when Adam struck Cain.

\subsubsection{Act 02 Scene 03 Page 119: Noah Strikes Japheth Again}
Use the same sound as in act 1 when Adam fights Cain.
When Yonah says ``No!'' stop the wind.

\subsubsection{Act 02 Scene 03 page 118: Noah Bangs Staff}
This is probably better done live.

\subsubsection{Act 02 Scene 03 Page 126: Rain Fades}
When Japheth says ``Look!'', start to fade the rain.
The release time should be 15 seconds, so the rain is gone
by the time Mama says ``An olive tree.''.

\subsection{Act 02 Scene 04 Page 130: Back on Land}
This scene is ``early one morning a few weeks later''.
Use the same background sound as for scene 1.

\subsection{After the Show}
When the last curtain has fallen, play some nice music as the audience
is leaving.

\section{Try it}
You can experience the show from the point of view of the sound
effects operator by running example 4 from sound\_effects\_player.
  
\section{Acknowledgments}
  
% Avoid a problem with the URL in the next section being broken across a page.
\newpage

\section{Licensing}
\label{section:Licensing}
As noted on the first page, this paper is licensed under a Creative
Commons Attribution-ShareAlike 4.0 International License.

The full text of the Creative Commons Attribution-ShareAlike 4.0
International license is at this web site:
\url{https://creativecommons.org/licenses/by-sa/4.0/legalcode}%
\embedfile[desc={Plaintext version of Creative Commons BY-SA 4.0 license},
  mimetype={text/plain}]{legalcode.txt}, and is embedded in this
PDF file.  What follows is a human-readable summary of it.

\subsection{You are free to:}
\begin{description}
\item[Share ---]copy and redistribute the material in any medium or format, and
\item[Adapt ---]remix, transform, and build upon the material
\end{description}
for any purpose, even commercially.  The licensor cannot revoke these
freedoms as long as you follow the license terms.
\subsection{Under the following terms:}
\begin{description}
\item[Attribution ---]You must give appropriate credit\footnote{If supplied,
  you must provide the name of the creator and attribution parties,
  a copyright notice, a license notice, a disclaimer notice, and a link
  to the material.}, provide a link to
  the license, and indicate if changes were made\footnote{You must indicate if
    you modified the material and retain an indication of previous
    modifications.}.  You may do so in any
  reasonable manner, but not in any that suggests the licensor endorses you
  or your use.
\item[ShareAlike ---]If you remix, transform, or build upon the material,
  you must distribute your contributions under the same
  license\footnote{You may also use any of the licenses listed as compatible
    at the following web site:
    \url{https://creativecommons.org/compatiblelicenses}}
  as the original.
\item[No additional restrictions ---]You may not apply legal terms or
  technological measures\footnote{The license prohibits application of
    effective technological measures, defined with reference to Article 11
    of the WIPO Copyright Treaty.}
  that legally restrict others from doing anything
  the license permits.
\end{description}
\subsection{Notices:}
\begin{itemize}
\item{You do not have to comply with the license for elements of the
  material in the public domain or where your use is permitted by an
  exception or limitation.}
\item{No warranties are given.  The license may not give you all of the
  permissions necessary for your intended use.  For example, other rights
  such as publicity, privacy or moral rights may limit how you use the
  material.}
\end{itemize}

% The bibliography contains URLs, and it is difficult to break URLs across
% a page boundary, particularly when there are footnotes.  To avoid this
% problem, put the bibliography on its own page.
\newpage

\bibliography{references}
\embedfile[desc={Bibliography},mimetype={text/plain}]{references.bib}

\end{document}
