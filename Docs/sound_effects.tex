% Main LaTeX source file for sound_effects.pdf

% Copyright © 2020 by John Sauter.
% Licensed under the Creative Commons Attribution-ShareAlike 4.0 International
% license.  See https://creativecommons.org/license/by-sa/4.0/.

\documentclass[letterpaper,twoside]{article}
\usepackage{fontspec}
\usepackage{amsmath}
\usepackage[english]{babel}
\usepackage{color}
\usepackage{multicol}
\usepackage{array}
\usepackage{longtable}
\usepackage{embedfile}
\usepackage{enumitem}
\usepackage[super]{nth}
\usepackage{fancyhdr}
\usepackage{xfrac}
\usepackage{fnpct}
\usepackage{siunitx}
\usepackage{graphics}
\usepackage{natbib}
\usepackage{hyperxmp}
\usepackage[tracking]{microtype}
\usepackage{hyphenat}
\usepackage{verbatim}
\usepackage[kpsewhich=true]{minted}
%\usepackage[usenames,dvipsnames]{color}

% Font choices: pick one.
% 1. Latin Modern matches Don Knuth's Computer Modern
%\setmainfont{Latin Modern Roman}[SmallCapsFont={Latin Modern Roman Caps}]
%\usepackage{unicode-math}

% 2. Libertine
\setmainfont[Ligatures={Common},Numbers=Proportional]{Linux Libertine O}
\setsansfont{Linux Biolinum O}
\usepackage{unicode-math}
\setmathfont[Scale=MatchUppercase]{libertinusmath-regular.otf}

% 3. Old Standard
%\setmainfont{Old Standard}[SmallCapsFont={Latin Modern Roman Caps}]

\usepackage[pdfencoding=unicode,pagebackref]{hyperref}
\bibliographystyle{plainnat}
\setcitestyle{numbers,square}
% Pages styles
%\setlength{\headheight}{22.5pt}
\pagestyle{fancy}
\fancyhead{}
\fancyhead[LE]{\thepage}
\fancyhead[CE]{{\scshape John Sauter}}
\fancyhead[CO]{{\scshape Sound Effects for Live Theatre}}
\fancyhead[RO]{\thepage}
\renewcommand{\headrulewidth}{0pt}
\fancyfoot{}
\setlength\tabcolsep{1mm}
\renewcommand\arraystretch{1.3}

\begin{document}
\embedfile[desc={Sound Effects for Live Theatre},
  mimetype={text/plain}]{sound_effects.tex}
\title{Sound Effects for Live Theatre\footnote{Copyright
    {\copyright} 2020 by John Sauter.
    This paper is made available under a
    Creative Commons Attribution-ShareAlike 4.0 International License.
    You can read a human-readable summary of the license at
    \url{https://creativecommons.org/licenses/by-sa/4.0}, which contains
    a link to the full text of the license.
    See also section \ref{section:Licensing} of this paper.}
}
\author{John Sauter\footnote{
    System Eyes Computer Store,
    20A Northwest Blvd.  Ste 345,
    Nashua, NH  03063-4066,
    e-mail: John\_Sauter@systemeyescomputerstore.com,
    telephone: (603) 424-1188}}
\hypersetup{unicode=true,
  pdfauthor={John Sauter},
  pdftitle={Sound Effects for Live Theatre},
  pdfsubject={Sound Effects for Live Theatre},
  pdfkeywords={Sound Effects, show\_control, sound\_effects\_player,
    Community Theatre},
  pdfcontactaddress={System Eyes Computer Store, 20A Northwest Blvd.  Ste 345},
  pdfcontactcity={Nashua},
  pdfcontactcountry={USA},
  pdfcontactemail={John\_Sauter@systemeyescomputerstore.com},
  pdfcontactphone={603-424-1188},
  pdfcontactpostcode={03063-4066},
  pdfcontactregion={New Hampshire},
  pdfcontacturl={https://www.systemeyescomputerstore.com},
  pdfcopyright={Copyright {\copyright} 2020 by John Sauter},
  pdfdate={2020-06-08},
  pdflicenseurl={https://creativecommons.org/licenses/by-sa/4.0},
  pdfmetalang={en-US}  
}
\date{2020-06-08}
\maketitle
\begin{abstract}
  How to produce sound effects for live theatrical performances
  on a small budget,
  for example for a Community Theatre production.
\end{abstract}
\begin{description}
\item[Keywords:]Sound Effects, Community Theatre.
\end{description}
  
%\tableofcontents
\newpage
\section{Motivation}
Low-budget live theatre uses recordings for music, rather than live
musicians, and leverages the music playing equipment (speakers,
amplifiers, a CD or similar music player) for sound effects.
The sound effects can be recorded on a CD or USB stick, and played
when needed during the performance.  This is adequate for background
sounds, but the timing is not good enough for a spot sound, such
as a slap.  In some cases these spot sounds can be made by the actors
or a backstage person using objects, but for others, such as a gun
shot, this is undesirable.  In addition, background sounds are
severely constrained: if an automobile must start its engine
while the background sounds are playing, you need a second CD
player.

Computers have become inexpensive enough in recent years that it
is reasonable to consider using one in place of the CD player
during a live performance.  With suitable software, a computer
can play both background and spot sounds, and mix them together
as needed.

I will describe how to use the sound\_effects\_player component
of the show\_control project to produce the sounds needed for
a live theatre production.

\section{What You Will Need}

To run the sound\_effects\_player you will need a computer running
the Fedora distribution of GNU/Linux.  The computer must also have
a sound card so it can output sound.  Sound\_effects\_player supports
up to eight channels of sound, but two are usually enough for Community
Theatre productions.

To get the sound\_effects\_player software, download the source tarball
from github.  It is a component of the show\_control project, at this
URL: \url{https://github.com/ShowControl/sound\_effects\_player}.
Sound\_effects\_player also requires libtime, another component of
show\_control.  Details are in the {\ttfamily .spec} file.
Alternatively, you can install the sound\_effects\_player
from copr, using john\_sauter / sound\_effects\_player.

Once you have the software installed you can look at the sample
configuration and the first three examples to learn about the
mechanics of the sound\_effects\_player and see how to make simple sounds.

\section{A Show with Complex Sound Effects}
Let us suppose that you have been asked to provide the sound effects
for a show with a challenging set of sound effects.  In addition to
the usual pre-show music and environment sounds during each scene,
there are also a variety of spot sounds that must be played
when an on-stage action happens, and some sounds must come from
the back of the audience.

The XML files developed in this paper are included with the distribution
of the sound\_effects\_player as example 4.

In order to make sounds come from behind the audience you have
borrowed two self-powered speakers on stands and positioned them
at the back of the theater.  You are using body microphones on
the actors.  These microphones go through the audio mixer and
the sound is sent equally to the two front speakers, so it seems
to the audience that it coming from the center of the stage.
The sound effects enter four channels of the audio mixer from
the computer running sound\_effects\_player.  In order to get
four sound channels out of the computer you are using a USB
sound device such as the M-Audio Fast Track Ultea 8R.
The audio mixer routes the four channels from the computer
directly to the four speakers.

The first thing we need in any sound project is the project file.



% Avoid a problem with the URL in the next section being broken across a page.
\newpage

\section{Licensing}
\label{section:Licensing}
As noted on the first page, this paper is licensed under a Creative
Commons Attribution-ShareAlike 4.0 International License.

The full text of the Creative Commons Attribution-ShareAlike 4.0
International license is at this web site:
\url{https://creativecommons.org/licenses/by-sa/4.0/legalcode}%
\embedfile[desc={Plaintext version of Creative Commons BY-SA 4.0 license},
  mimetype={text/plain}]{legalcode.txt}, and is embedded in this
PDF file.  What follows is a human-readable summary of it.

\subsection{You are free to:}
\begin{description}
\item[Share ---]copy and redistribute the material in any medium or format, and
\item[Adapt ---]remix, transform, and build upon the material
\end{description}
for any purpose, even commercially.  The licensor cannot revoke these
freedoms as long as you follow the license terms.
\subsection{Under the following terms:}
\begin{description}
\item[Attribution ---]You must give appropriate credit\footnote{If supplied,
  you must provide the name of the creator and attribution parties,
  a copyright notice, a license notice, a disclaimer notice, and a link
  to the material.}, provide a link to
  the license, and indicate if changes were made\footnote{You must indicate if
    you modified the material and retain an indication of previous
    modifications.}.  You may do so in any
  reasonable manner, but not in any that suggests the licensor endorses you
  or your use.
\item[ShareAlike ---]If you remix, transform, or build upon the material,
  you must distribute your contributions under the same
  license\footnote{You may also use any of the licenses listed as compatible
    at the following web site:
    \url{https://creativecommons.org/compatiblelicenses}}
  as the original.
\item[No additional restrictions ---]You may not apply legal terms or
  technological measures\footnote{The license prohibits application of
    effective technological measures, defined with reference to Article 11
    of the WIPO Copyright Treaty.}
  that legally restrict others from doing anything
  the license permits.
\end{description}
\subsection{Notices:}
\begin{itemize}
\item{You do not have to comply with the license for elements of the
  material in the public domain or where your use is permitted by an
  exception or limitation.}
\item{No warranties are given.  The license may not give you all of the
  permissions necessary for your intended use.  For example, other rights
  such as publicity, privacy or moral rights may limit how you use the
  material.}
\end{itemize}

% The bibliography contains URLs, and it is difficult to break URLs across
% a page boundary, particularly when there are footnotes.  To avoid this
% problem, put the bibliography on its own page.
\newpage

\bibliography{references}
\embedfile[desc={Bibliography},mimetype={text/plain}]{references.bib}

\end{document}
